\chapter{Robot Overview}\label{ch:overview}
This chapter goes over the components of the robot and how they interact with one another at a high level. 
For more detailed information about a specific component and examples running a component through ROS see chapter ~\ref{ch:componentspecifics}.

\section{Sensors}
There are three primary types of sensors aboard PoliV2: lidar, sonar and RGB-D.
Two lidar sensors, a Hokuyo 30EW at the front, and a Hokuyo 10LX at the rear, are used for primary localization and obstacle detection.\\

A Maxbotix 1230-EZ sonar sensor is used for additional obstacle detection of surfaces not detectable by 2D laser, e.g. glass or stairs.\\

Finally an Astra RGB-D camera is used for general perception. 
While possible to fuse depth data into localization and obstacle detection, this is not done for the Astra whose primary perception is used for manipulation of objects.

\section{Manipulation}
Robotic manipulation is accomplished with a custom Kinova Jaco2 7DOF robotic manipulator fitted with a Weiss WSG-32 gripper with force-sensing finger tips. Both the arm and gripper are exposed to the ROS network.\\

The robot also has two additional degrees of freedom in the pan-tilt construction on the head and neck of the robot.
This is accomplished with a Dynamixel X Series servo for the pan degree, and a Maxon motor with planetary gearbox for the tilt degree.

\section{Platform}
The robot's mobility is provided via the Segway RMP 110 platform, which offers two wheeled differential drive. 
The acceleration and velocity capabilities are quite high and are typically scaled back via soft limits.

\subsection{Platform Power}
Power is provided in through two means: propulsion power and auxiliary power. 
Both are provided and charged through the RMP platform. Propulsion is strictly for internal use of the base. \\

The auxiliary power supplies the computers, arm, pillar, and other components. 
As it stands, this allows for around 4.5 hours of runtime while disconnected from the charger. 
If running the robot connected to the charger, expect about 12 hours of run time before the auxiliary battery is depleted. \\

Charging the batteries takes 8 to 10 hours due to the chemistry of the batteries. 
See details about charging in Section \ref{sec:charging_robot}.


\section{Computers}
There are two computers present, each an Intel NUC \textit{NUC6i7KYK} featuring a 2.6 GHz processor, 32 Gb of RAM, and Iris 580 graphics. 
Both PCs are networked locally and have access to network resources. 
Additionally, each PC is responsible for a number of locally connected USB devices. 
More on this in Chapter \ref{sec:network_configuration}.\\

The PCs are enumerated with the robot as a base name, making them \texttt{poli1} and \texttt{poli2}. 
There is a defacto user account, \texttt{poli} on each computer.

\subsection{ROS Environment}
All computers run Ubuntu 16.04 LTS and ROS Kinetic.

\section{Glossary of terms}
In the following chapters, specific terms are used to refer to the robot and its components. 
Please refer below to help disambiguate.

\begin{itemize}
\item PoliV2/Poli2 - The name for an instance of the whole robot platform
\item poli1 - The name for a PC on the PoliV2 platform, specifically the PC mounted on the base.
\item poli2 - The name for a PC on the PoliV2 platform, specifically the PC mounted on the torso.
\item poli2 high level repository - The name for the organizational repository for the PoliV2 platform. Can be found at \href{https://github.com/si-machines/poli2}{https://github.com/si-machines/poli2}
\end{itemize}

