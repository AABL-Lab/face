\chapter{Setting up a New Robot}\label{ch:setting_up_new_robot}
This section contains all information pertaining to setting up and configuring the components of the robot as they come from the factory. 
These changes are strictly software configuration necessary for the components to communicate and would only need to be carried out once per robot.

\section{Setting up Network Components}
The Segway RMP base is set in the 10.66.171.x sub net. This requires us to change all other networked devices (router, lasers, computers) to accept being addressed in this range. For simplicity, we try to keep all IP addresses consistent across all robots. Reference Figure \ref{fig:network_map} when setting the IPs of the various components.

\subsection{Setting up network hosts}
In order to resolve network names and communicate over a common network, both \textbf{poli1} and \textbf{poli2} need each others names resolved. See example \ref{ex:network_hosts}

\begin{example}{Configure network hosts}\label{ex:network_hosts}
  \begin{itemize}
    \item On \textbf{poli1}
    \item Edit \texttt{/etc/hosts} with \texttt{sudo}
    \item Add \texttt{10.66.171.24 poli2}
    \item Save and exit
    \item On \textbf{poli2}
    \item Edit \texttt{/etc/hosts} with \texttt{sudo}
    \item Add \texttt{10.66.171.91 poli1}
    \item Save and exit
  \end{itemize}
\end{example}

\subsection{Setting up the connections in Ubuntu}
Example \ref{ex:network_manager} is required to simultaneously use Internet via WiFi and LAN via ethernet. 
This manually manages the ethernet connection that allows connection to the LAN and prevents Ubuntu from using it as an outbound route.

\begin{example}{Configuring Ubuntu's network manager}\label{ex:network_manager}
	 \texttt{ auto lo \\
 iface lo inet loopback \\
 auto eno1 \\
 iface eno1 inet static \\
 address 10.66.171.x\\
 netmask 255.255.255.0}
   \begin{itemize}
     \item Edit \texttt{etc/network/interfaces} file with \texttt{sudo}
     \item Fill in the file with the above codeblock
     \item Change the IP address after \texttt{address} to match the IP of the computer whose file you're editing. 
     \item Restart computer.
   \end{itemize}
\end{example}

\subsection{Setting up network connections for the router}
When setting up the main router, change the router's IP address to \texttt{10.66.171.1}. \\

While you're there, change the network name to something appropriate (SSID = PoliV2-N) where N == a number that won't cause collisions with existing robot networks. \\

Finally, go to the DHCP client table. Here you will be able to bind IP addresses of important components. You will need to bind the IPs of the wireless bridge, \texttt{poli1} and \texttt{poli2} at a minimum. \\

\subsection{Change the IP Addresses of the lasers}
The default IP address of the Hokukyo lasers are 192.168.0.10 \\

You can find a link to the Windows IP changing tool provided by Hokuyo in Section \ref{sec:sensor_data_sheets}. 
The ROS \texttt{urg\_node}, which is what we use to fetch data from the lasers, also has a node called \texttt{change\_ip\_address}. 
This has not been tested by our group, but may work in place of the Windows tool. \\

Change the laser IPs to the appropriate value. 
Ensure the correct laser gets the expected IP address.

\subsection{Configure the Gripper's IP}
Next, configure the Gripper's IP to be statically 10.66.171.21. 
This can be accomplished through the gripper's web interface at its default IP address, \texttt{192.168.1.20}. 

This should be done outside the immediate network, though, if you've already changed the IP range of the router.



\subsection{Configure the Gripper's Wireless Bridge}
Finally, configure the wireless bridge to be a client of the router and reset its IP range.
The bridge has a default IP address of \texttt{192.168.1.1}.

\begin{example}{Reconfiguring the wireless bridge}
  \begin{itemize}
    \item Connect to bridge's LAN interface
    \item Navigate to \texttt{Quick Setup} on the left navigation panel
    \item Click Next
    \item Select \textbf{Client} under `At Home'
    \item Click Next
    \item Select the local router SSID for the robot you're configuring
    \item Enter the correct security credentials
    \item Finish. Results in a reboot.
    \item Select \textbf{Network} on the left navigation panel
    \item Select \textbf{Static IP} for type
    \item Enter \textbf{10.66.171.2} for IP address
    \item Enter \textbf{10.66.171.1} for the gateway
  \end{itemize}
\end{example}

